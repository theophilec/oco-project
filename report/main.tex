\documentclass[twoside,11pt]{article}

% \usepackage[preprint, abbrvbib]{jmlr2e}
\usepackage[preprint,abbrvbib]{jmlr2e}
\usepackage[utf8]{inputenc}
\usepackage{tikz-cd}
\jmlrheading{1}{2020}{1-38}{4/00}{10/00}{cantelobre21}{Alexandre Thomas, Théophile Cantelobre}
\ShortHeadings{PAC-Bayesian Structured Prediction with ILE}{Thomas Cantelobre}
\firstpageno{1}
\usepackage{color}
\usepackage{diagbox}
% MATH SYMBOLS AND ENVIRONMENTS
\usepackage{latexsym}
\usepackage{amsmath}
\usepackage{amsfonts}
\usepackage[mathcal]{eucal}
\usepackage{multirow}
\usepackage{xspace}
% ALGORITHMS ENVIRONMENT
\usepackage[ruled,vlined,linesnumbered]{algorithm2e}
\usepackage{adjustbox}
% USEFUL COMMANDS
\newcommand{\paren}[1]{\left( #1 \right)}
\newcommand{\croch}[1]{\left[\, #1 \,\right]}
\newcommand{\acc}[1]{\left\{ #1 \right\}}
\newcommand{\abs}[1]{\left| #1 \right|}
\newcommand{\norm}[1]{\left\Vert #1 \right\Vert}
\newcommand{\todo}[1]{\textbf{\color{red}{[TODO: #1]}}}
\newcommand{\ok}{\textbf{\color{red}{[OK]}}}
\newcommand{\expect}[1]{\mathop{\mathbb{E}}_{#1}}
\newcommand{\var}[1]{\mathop{\mathbb{V}}_{#1}}
\newcommand{\note}[1]{{\textbf{\color{red}#1}}}

\newcommand{\ind}{\perp\!\!\!\!\perp}
\newcommand{\grad}{\nabla}
\DeclareMathOperator{\Tr}{Tr}
\DeclareMathOperator{\Cov}{Cov}

\makeatother

% THEOREM STYLES
\usepackage{amsthm}
\usepackage{cleveref}
\newcommand{\BlackBox}{\rule{1.5ex}{1.5ex}}  % end of proof
\renewenvironment{proof}{\par\noindent{\bf Proof\ }}{\hfill\BlackBox\\[2mm]}
\newtheorem{example}{Example} 
\newtheorem{theorem}{Theorem}
\newtheorem{lemma}[theorem]{Lemma} 
\newtheorem{proposition}[theorem]{Proposition} 
\newtheorem{remark}[theorem]{Remark}
\newtheorem{corollary}[theorem]{Corollary}
\newtheorem{definition}[theorem]{Definition}
\newtheorem{conjecture}[theorem]{Conjecture}
\newtheorem{axiom}[theorem]{Axiom}

\newtheorem{lemme}{Lemma}
\newtheorem{prop}[theorem]{Proposition}
\newtheorem{examples}{Examples}
\newtheorem{problem}{Problem}
\newtheorem{warning}{\textcolor{red}{Warning}}[theorem]

% CLEAN REFERENCE HANDLING
\crefname{problem}{Problem}{Problems}
\crefname{theorem}{Theorem}{Theorems}
\crefname{prop}{Proposition}{Propositions}
\crefname{equation}{Eq.}{Eqs.}
\crefname{figure}{Figure}{Figures}
\crefname{section}{Section}{Sections}
\crefname{table}{Table}{Tables}
\crefname{lemme}{Lemma}{Lemmas}
\crefname{corollary}{Corollary}{Corollaries}
\crefname{example}{Example}{Examples}
\crefname{appendix}{Appendix}{Appendixes}
\crefname{remark}{Remark}{Remark}
%\crefname{alg}{Algorithm}{Algorithm}



\begin{document}

\title{Hogwild} 
\author{%
\name Alexandre Thomas \\
\addr Mines ParisTech -- PSL, Sorbonne Université
\AND
\name Théophile Cantelobre\\
\addr Mines ParisTech -- PSL, Sorbonne Université
}

\maketitle


\todo{Convert to OCO course notation?}

\section{Formalism}
\subsection{Goal} The goal of Hogwild is to minimize a function $f: \mathbb R^n \rightarrow \mathbb R$ which decomposes as:
\begin{equation*}
    f(x) = \sum_{e\in E}f_e(x_e)
\end{equation*}

where:
\begin{itemize}
    \item $e\subset [n]$ a small subset of $[n]$,
    \item $x_e\in\mathbb R^{\vert e \vert}$, the values of the vector $x$ on its coordinates indexed by $e$.
\end{itemize}

\subsection{Sparse SVM problem formulation}
\paragraph{Dataset}
Given $m = \vert E \vert$ pairs of data points  $E = \lbrace (a_1, b_1), \ldots, (a_m, b_m)\rbrace\in \left(\mathbb R^n \times \lbrace 0, 1\rbrace\right)^m$.

\paragraph{SVM loss} Given $\lambda > 0$, learning an SVM is equivalent to the following optimization problem:
\begin{problem}[SVM]\label{prob:svm}
    \begin{equation*}
        \min_{x\in\mathbb R^n}\sum_{\alpha \in E}\max(1- y_\alpha x^Ta_\alpha, 0) + \lambda\norm{x}_2^2
    \end{equation*}
\end{problem}

\paragraph{SVM to Sparse-SVM}

Hypothesis: $a_\alpha$ are very sparse.

Given $\alpha \in [\vert E \vert]$, let $e_\alpha$ denote the non-zero components of $a_\alpha$. Conversely, given $u\in[n]$, let $d_u$ denote the number of training examples $a$ for which $a_u$ is non-zero.

Given $\alpha \in [\vert E\vert]$, then 

\begin{equation}
    \max(1- b_\alpha x^Ta_\alpha, 0) = \max(1- b_\alpha x_\alpha^Ta_\alpha, 0)
\end{equation}

and

\begin{proposition}
    Assume $\forall u \in[n], d_u > 0$. Then,
    \begin{equation}
        \norm{x}^2 = \sum_{\alpha \in E}\sum_{u\in e_\alpha}\frac{x_u^2}{d_u}
    \end{equation}
\end{proposition}

\begin{proof}
    Notice that
    \begin{equation*}
        \sum_{\alpha \in E} \norm{x_{e_\alpha}}^2 = \sum_{u=1}^n x_u^2d_u.
    \end{equation*}
\end{proof}
So, we can rewrite \cref{prob:svm} as a \emph{sparse SVM} problem as in \cref{prob:sparse-svm}:
\begin{problem}[Sparse SVM]\label{prob:sparse-svm}
    \begin{equation*}
        \min_{x\in\mathbb R^n}\sum_{\alpha \in E}\underbrace{\max(1- b_\alpha x_\alpha^Ta_\alpha, 0) + \lambda\sum_{u\in e_\alpha}\frac{x_u^2}{d_u}}_\text{$f_\alpha(x_\alpha)$}
    \end{equation*}
\end{problem}

\subsection{Quantifying sparsity on MNIST}
\todo{Define $\Omega$, $\Delta$ and $\rho$.}

\section{\texttt{Hogwild !} algorithm}
\subsection{Computation model}
Shared memory model with $p$ processors. More precisely:
\begin{itemize}
    \item $x$ can be read by any processor.
    \item The component-wise addition operator is atomic. In other words, a single component of $x$ can be modified by adding a scalar without locking.
\end{itemize}

\subsection{Algorithm description}
Taking a lock-free approach, each processor, randomly samples $e$ from $E$ uniformly at random (the uniform distribution over $E$ is denoted $\mathcal E$). It then reads $\hat x_e \leftarrow x_e$ from the shared buffer. The gradient of $f_e$ is computed using $\hat x_e$, denoted $G_e(\hat x_e)$. Finally, the gradient update step is carried out component by component using atomic addition of $\gamma \delta_v^TG_e(\hat x_e)$ for every $v \in e$. Here, $\delta_v$ is the one-hot encoding of $v$ over
$\lbrace{1, \ldots, n\rbrace}$. A \texttt{Hogwild!} step for an individual processor is given in \Cref{alg:hogwild-step}.


\begin{algorithm}[ht]
\SetAlgoLined
\SetKwInOut{Input}{Input}
\SetKwInOut{Parameter}{Parameters}
%\Input{Initial posterior mean $W_0$.}
\Parameter{Step-size $\gamma$.}
Sample $e\sim\mathcal E$ uniformly over $E$.\\
Read buffer $\hat x_e \leftarrow x_e$.\\
Compute $G_e(\hat x_e)$.\\
\For{$v\in e$}{
    Atomic addition $x_v \leftarrow x_v - \gamma \delta_v^TG_e(\hat x_e)$ (in buffer).\\
}
\caption{\texttt{Hogwild!} step for an individual processor.}
\label{alg:hogwild-step}
\end{algorithm}

\subsection{Convergence analysis}
\todo{}

\subsection{Runtime analysis}
\todo{}

\section{Implementation}
We implemented \texttt{Hogwild!} in \texttt{Python} relying on the \texttt{multiprocessing} library to handle the parallel programming aspects of the algorithm. Note that the experiments in the paper are based on a \texttt{C} implementation which was not released. 

In this section, we explain the architecture of the code at a high-level. Although we include code snippets here, our source code is available at: \todo{github address}. Our code is architectured to interface with the code for the other algorithms presented in this report\footnote{See, for example, \href{https://github.com/srome/sklearn-hogwild}{this Github repository} for a \texttt{scikit-learn}-oriented implementation.}.



\section{Experimental results}

\newpage
Remove this later: \cite{ciliberto_general_2020}
% \acks{Acknowledgments go here.}

% \appendix

\vskip 0.2in
\newpage
\bibliography{biblio}

\end{document}
